\documentclass{beamer}
%\documentclass[12pt]{cfpbpresentation}

\AtBeginSection[]{
  \begin{frame}
  \vfill
  \centering
  \begin{beamercolorbox}[sep=8pt,center,shadow=true,rounded=true]{title}
    \usebeamerfont{title}\insertsectionhead\par%
  \end{beamercolorbox}
  \vfill
  \end{frame}
}

\usepackage{graphicx}
\usepackage{caption}
%\usepackage{subcaption}
\usepackage{amsmath}
\usepackage{setspace}
\usepackage{amssymb}
\usepackage{hyperref}
\usepackage{colortbl,color}
\usepackage{cancel}
\usepackage{multimedia}
\usepackage{color}
\usepackage{tikz}
\usepackage[english]{babel}
\usepackage[flushleft]{threeparttable}
\usepackage[style=authoryear]{biblatex}
\usepackage[normalem]{ulem}

\usepackage{array}
\usepackage{multicol}
\usepackage{multirow}
\usepackage{hhline}

\graphicspath{ {../} }

%\usetheme{Amsterdam}
\usetheme{Singapore}
\setbeamertemplate{navigation symbols}{}
%\useoutertheme{infolines} 


\newcommand{\sigsq}{\sigma^2}
\newcommand{\redpause}{\addtocounter{beamerpauses}{-1}\pause\color<+>{red}}
\newcommand{\ft[1]}{\begin{frame} \frametitle{#1}}
\newcommand{\ftl}[2]{\begin{frame}[label=#1] \frametitle{#2}}
\newcommand{\nframe[2]}{\begin{frame}\frametitle{#1} #2 \end{frame}}

\newcommand{\citepos}[1]{\citeauthor{#1}' (\citeyear{#1})}

\setbeamersize{text margin left=1em,text margin right=1em}
\mathchardef\mhyphen="2D % Define a "math hyphen"

\begin{document}

\title[OOP, Python, and HARK] % (optional, use only with long paper titles)
{Object-Oriented Programming, Python, and HARK}

\author
{David Low (CFPB)}

\institute[CFPB]
%{CFPB}

\date[April 26, 2017]
{April 26, 2017}

%\date[\today]
%{\today}

%%%%%%%%%%%%%%%%%%%%%%%%%%%%%%%%%%%%%%%%%%%%%%%%%%%%%%%
%%%%%%%%%%%%%%%%%%%%%%%%%%%%%%%%%%%%%%%%%%%%%%%%%%%%%%%

\begin{frame}
  \titlepage
\end{frame}


%%%%%%%%%%%%%%%%%%%%%%%%%%%%%%%%%%%%%%%%%%%%%%%%%%%%%%%%
%%%%%%%%%%%%%%%%%%%%%%%%%%%%%%%%%%%%%%%%%%%%%%%%%%%%%%%%


\ft[What Is Object-Oriented Programming (OOP)?]

Traditional programming is \emph{procedure}-oriented.  
\begin{itemize}
\item Basically, do things as you think of them.
\item Focus is on the \emph{procedure} necessary to accomplish goals.
\item Most immediately intuitive way to program (or do anything.)  
\end{itemize}

\

Procedure-oriented pseudo-code to grow tomatoes and strawberries:
\begin{enumerate}
\item Plant strawberry seed
\item Water strawberry
\item Pick strawberries in June
\item Plant tomato seed
\item Water tomato
\item Pick tomatoes in August
\end{enumerate}

\end{frame}

%%%%%%%%%%%%%%%%%%%%%%%%%%%%%%%%%%%%%%%%%%%%%%%%%%%%%%%%

\ft[What Is Object-Oriented Programming?]

Object-oriented programming is (ahem) \emph{object}-oriented.  
\begin{itemize}
\item Focus on \emph{objects} (abstract structures) necessary to accomplish goals.
\item Less immediately intuitive... but often far more useful.
\end{itemize}

\

It is useful in the same way abstraction is useful in any other field
	\begin{itemize}
	\item History: what do revolutions have in common?
		\begin{itemize}
		\item Weak government, fractured elite, don't start off as revolutions, etc.
		\end{itemize}
	\item Math: what do metric spaces have in common?
		\begin{itemize}
		\item Defined by a metric, which in turn defines open and closed sets
		\end{itemize}
	\item Programming: what should chunks of my code have in common?
		\begin{itemize}
		\item Helps clarify thinking, like abstraction in other fields
		\item Benefits unique to programming: avoid repetition; cleaner code
		\end{itemize}
	\end{itemize}


\end{frame}


%%%%%%%%%%%%%%%%%%%%%%%%%%%%%%%%%%%%%%%%%%%%%%%%%%%%%%%%

\ft[What Is Object-Oriented Programming?]

In our example, tomatoes and strawberries are both fruit plants.

\

Object-oriented pseudo-code:
\begin{enumerate}
\item tomato $=$ FruitPlant(harvest\_time = August)
\item tomato.plantSeed()
\item tomato.water()
\item tomato.pickFruit()
\item strawberry $=$ FruitPlant(harvest\_time = June)
\item strawberry.plantSeed()
\item strawberry.water()
\item strawberry.pickFruit()
\end{enumerate}

\

Fruit plants ``know'' how (and when!) to plant, water, and pick themselves.

\end{frame}


%%%%%%%%%%%%%%%%%%%%%%%%%%%%%%%%%%%%%%%%%%%%%%%%%%%%%%%%

\ft[Why Object-Oriented Programming?]

Clarifies your thinking
	\begin{itemize}
	\item Recognizing similarities and differences between tomatoes and strawberries might help you think about blueberries or Redwood trees or Venus fly traps, too.
	\item Higher-level thinking is much more fun!
	\end{itemize}

\

Lets you avoid repetition, which is hugely important in programming
	\begin{itemize}
	\item Less effort
	\item Cleaner code
	\item Easier to debug
	\end{itemize}

\

These things are nice for small projects.  They are \emph{amazing} for big projects.

\

Still, this might seem abstract -- it did to me at first.  

Becomes more obvious with experience.

\end{frame}

%%%%%%%%%%%%%%%%%%%%%%%%%%%%%%%%%%%%%%%%%%%%%%%%%%%%%%%%

\ft[Why Python? (This one's easy!)]

Main reason to learn Python over any other comparably high-level scientific computing language (Matlab; Julia) is OOP

\

\

Python provides amazing support for OOP (though it also allows procedure-oriented programming)

\

\

Python makes it extremely easy to write good, clean, clear, reusable code
\begin{itemize}
\item These qualities are especially important for large projects (e.g. code for your job market paper, or HARK)
\end{itemize}


\end{frame}


%%%%%%%%%%%%%%%%%%%%%%%%%%%%%%%%%%%%%%%%%%%%%%%%%%%%%%%%

\ft[Basic OOP (in Python)]

A \emph{class} is an abstract grouping, members of which have stuff in common 
\begin{itemize}
\item E.g., a fruit plant
\end{itemize}

\

An \emph{instance} is a specific implementation of a class 
\begin{itemize}
\item E.g., a tomato or strawberry
\end{itemize}


\

An \emph{attribute} is just an object attached to a class instance.
	\begin{itemize}
	\item Usually, but not always, instances of a class have the same attributes
	\item E.g. whether or not the plant is watered
	\end{itemize}

\

A \emph{method} is a function attached to an instance that also operates on it
	\begin{itemize}
	\item This sounds complex.  Really, just how instances  ``know'' how to do important things to themselves.  E.g. tomato.pickFruit()
	\end{itemize}


\end{frame}

%%%%%%%%%%%%%%%%%%%%%%%%%%%%%%%%%%%%%%%%%%%%%%%%%%%%%%%%

\ft[Inheritance]

How should objects be grouped together into classes?  Often, they will have some things in common, but not others.

\

Good way to deal with this is called \emph{inheritance}:
	\begin{itemize}
	\item Broad classes; instances share a few things in common (e.g. plants)
	\item Narrow classes; instances have more in common (e.g. fruit plants)
	\item Classes can \emph{inherit} from other classes; avoid redefining attributes
		\begin{itemize}
		\item So narrow classes can \emph{inherit} from broader ones
		\item E.g. fruit plants are plants
		\end{itemize}
	\end{itemize}

\

Classes can inherit from multiple other classes
	\begin{itemize}
	\item Called \emph{multiple inheritance}
	\item Fruit trees are fruit plants, but they are also things to climb.
	\end{itemize}

%[See code]

\end{frame}



%%%%%%%%%%%%%%%%%%%%%%%%%%%%%%%%%%%%%%%%%%%%%%%%%%%%%%%%

\ft[Code Exploration]


\end{frame}



%%%%%%%%%%%%%%%%%%%%%%%%%%%%%%%%%%%%%%%%%%%%%%%%%%%%%%%%

\ft[Now on to HARK!]

[But first, a 10 minute break.]

\end{frame}


%%%%%%%%%%%%%%%%%%%%%%%%%%%%%%%%%%%%%%%%%%%%%%%%%%%%%%%%

\ft[OOP and HARK]

Whole idea behind HARK: stop solving models from scratch
	\begin{itemize}
	\item If the model has been solved before, use that code
	\item If the model hasn't been solved before, take code for a similar model, and modify it only where necessary
	\end{itemize}

\

This is conceptually almost identical to OOP

\

How?  HARK thinks of types of agents as classes
	\begin{itemize}
	\item Matt White will give a lot more detail when he visits
	\item Agents in HARK ``know'' how to solve their model, through the $solve()$ method
	\item If desired, they can then interact with other agents with potentially different parameters or problems
	\end{itemize}


\end{frame}

%%%%%%%%%%%%%%%%%%%%%%%%%%%%%%%%%%%%%%%%%%%%%%%%%%%%%%%%

\ft[HARK and OOP 1]

Example 1: Import the model, set a few parameters, and solve it

\

E.g.  MPC out of credit vs MPC out of temporary income

\

Currently two other demos in HARK:
\begin{enumerate}
\item Precautionary saving and the Great Recession
\item Precautionary saving and Chinese growth
\end{enumerate}

\end{frame}

%%%%%%%%%%%%%%%%%%%%%%%%%%%%%%%%%%%%%%%%%%%%%%%%%%%%%%%%

\ft[HARK and OOP 2]

Example 2: Import one model, change it a little bit, and solve new model.

\

HARK includes the ``standard'' consumption-savings model.  

It also includes two extensions:
\begin{enumerate}
\item $R^{debt} \neq R^{savings}$
\item Multiplicative utility shocks
\end{enumerate}

What about both at the same time?

\

KinkyPrefConsumerType in ConsPrefShockModel.py, combination of:

\begin{enumerate}
\item KinkedRConsumerType from ConsIndShockModel.py
\item PrefShockConsumerType, from ConsPrefShockModel.py
\end{enumerate}

\

Imagine doing this without OOP!

\end{frame}


%%%%%%%%%%%%%%%%%%%%%%%%%%%%%%%%%%%%%%%%%%%%%%%%%%%%%%%%

\ft[Conclusion]

OOP, Python, and HARK are all great.

\

\

You should learn them!


\end{frame}



\end{document}