\documentclass[public]{beamer}
\usepackage{ifthen}
\usepackage{econtexShortcuts}
\usepackage{verbatim}

\usetheme{Madrid}
\definecolor{orange}{HTML}{FF7F00}
\hypersetup{colorlinks,linkcolor=,urlcolor=orange}
%\usepackage{econtexSetup}

% Redefine footer from
% http://tex.stackexchange.com/questions/83048/change-the-contents-of-footline-in-a-beamer-presentation
\setbeamertemplate{footline}
{
  \leavevmode%
  \hbox{%
  \begin{beamercolorbox}[wd=.333333\paperwidth,ht=2.25ex,dp=1ex,center]{author in head/foot}%
    \usebeamerfont{author in head/foot}\insertsection
  \end{beamercolorbox}%
  \begin{beamercolorbox}[wd=.333333\paperwidth,ht=2.25ex,dp=1ex,center]{title in head/foot}%
    \usebeamerfont{title in head/foot}\insertsubsection
  \end{beamercolorbox}%
  \begin{beamercolorbox}[wd=.333333\paperwidth,ht=2.25ex,dp=1ex,right]{date in head/foot}%
    \usebeamerfont{date in head/foot}\insertshortdate{}\hspace*{2em}
    \insertframenumber{} / \inserttotalframenumber\hspace*{2ex} 
  \end{beamercolorbox}}%
  \vskip0pt%
}
\makeatother

\providecommand{\ARK}{\href{http://github.com/econ-ark}{github.com/econ-ark}}

\providecommand{\CDC}{\texttt{CDC}}
\providecommand{\NMP}{\texttt{NMP}}
\providecommand{\MNW}{\texttt{MNW}}
\providecommand{\DCL}{\texttt{DCL}}
\providecommand{\JXY}{\texttt{JXY}}
\providecommand{\AMK}{\texttt{AMK}}


\beamerdefaultoverlayspecification{<+->}


\usepackage{natbib}
\begin{document}
\title[\ARK]{Introducing  the Computational Economics \\ ``Algorithmic Repository and toolKit'' \\
\href{http://github.com/econ-ark}{github.com/econ-ark}}

\author[Chris Carroll]{Presentation by Chris Carroll at Conference on \\ 
{Computation in Economics and Finance (CEF), Bordeaux}}
\date{June, 2016}



\begin{frame}
  \titlepage
\end{frame}

\section{Who, What, Where, When, Why, How}
\subsection{What}

\begin{frame}
\frametitle{What Is It?}

State-of-the-art set of tools for:

\begin{enumerate} \pause
\item Simulating behavior of populations of agents
\item Solving dynamic stochastic optimization problems
\begin{itemize}
\item Particularly adapted for Bellman problems with `kinks' and quirks
\end{itemize}
\end{enumerate}
\end{frame}

\begin{frame}
\frametitle{What Is It Good For?}

\begin{itemize}
\item Heterogeneous Agent Macro Models
\begin{itemize}
\item Original name: {\bf H}eterogeneous {\bf A}gent {\bf R}esources and tool{\bf K}it
\item HARK!
\end{itemize}
\item Structural Micro Models (e.g., labor, health)
\item IO models with optimizing consumers and firms
\item {\bf N}ot {\bf O}nly {\bf A}bout {\bf H}eterogeneous {\bf s}tuff ... 
\item ... {\bf A}lgorithmic {\bf R}esources and tool{\bf K}it 
\item :-)
\begin{itemize}
\item Unlike Noah's, our ARK can hold more than two of each kind!
\item Ultimate goal: Get examples on the ARK of all types of animal (model)
\end{itemize}
\end{itemize}


\end{frame}

\subsection{Where}
\begin{frame}
\frametitle{Where Is It?}

\begin{center}
{\ARK} is the project's home
\end{center}

\begin{enumerate}
\item Get a GitHub account.  Then, options to access are
\begin{itemize}
\item Install GitHub Desktop App
\item Install `git' command-line tool (if you're hard-core)
\end{itemize}
\item \texttt{{\ARK}/HARK} is a ``public repo'' 
\begin{itemize}
\item Contains all existing code
\end{itemize}
\item \texttt{{\ARK}/HARK/Documentation/NARK.pdf}
\begin{itemize}
\item Describes variable naming conventions for easy workflow: 
\item LaTeX object definitions correspond to HARK definitions 
\end{itemize}
\item Similar structure will be used for future contributions
\begin{itemize}
\item {\bf A}gent {\bf A}rchive {\bf R}epository {\bf D}eposit {\bf V}ehicle for ARK?
\end{itemize}
\item Instructions for cloning are in the README.txt
\item You get the whole codebase under the Apache license 
\begin{itemize}
\item Basically, no limitations on use
\item But, please credit us, and participate in discussions
\end{itemize}
\end{enumerate}

\end{frame}

\subsection{Who}
\begin{frame}
\frametitle{Who Has Produced It?}

\begin{footnotesize}
\begin{center}

\begin{tabular}{lll}
Name & TLA & Affiliation % & Contact
\\ \hline \hline {\it Christopher D Carroll} & \texttt{{\CDC}} & JHU, CFPB % & \href{mailto:ccarroll@llorracc.org}{ccarroll@llorracc.org}
\\ {\it David C Low} & \texttt{{\DCL}} & CFPB % & \href{mailto:ccarroll@llorracc.org}{ccarroll@llorracc.org}
\\ {\it Nathan M Palmer} & \texttt{{\NMP}} & OFR % & \href{malito:nathan.m.palmer@gmail.com}{nathan.m.palmer@gmail.com}
\\ {\it Matthew N White} & \texttt{{\MNW}} & UDel, CFPB % & \href{mailto:mnwhite@gmail.com}{mnwhite@gmail.com}
\\ \hline {\it Alex Kaufman} & \texttt{{\AMK}} & CFPB $\rightarrow$ ? (Alcatraz?)  %& \href{mailto:alexander.kaufman@cfpb.gov}{No Fixed Address; Expected: Alcatraz}
\\ {\it Jiaxiong Yao} & \texttt{JXY} & JHU $\rightarrow$ IMF  %& \href{mailto:alexander.kaufman@cfpb.gov}{No Fixed Address; Expected: Alcatraz}
\end{tabular}
\end{center}


Nothing herein may be interpreted as reflecing opinions of 
\begin{center}
\begin{tabular}{rcl}
 CFPB & - & United States Consumer Financial Protection Bureau
\\ JHU & - & Johns Hopkins University
\\ IMF & - & International Monetary Fund
\\ OFR & - & Office of Financial Research, U.S.\ Treasury
\\ UDel & - & University of Delaware
\end{tabular}

\end{center}


\end{footnotesize}
\end{frame}

\begin{frame}
\frametitle{Major credit goes to CFPB - a 21st Century Regulator!}

\begin{itemize}
\item Hired {\CDC} as Chief Economist with this as a key priority
\item Hired {\NMP} as intern to get started
\item Hired {\MNW} as Visiting Scholar to work on it
\item Hired {\DCL} as new economist last year
\item Hired {\AMK} as RA
\end{itemize}
\end{frame}


\begin{frame}
\frametitle{Organization Going Forward}

Standard Github tools, esp:
\begin{itemize}
\item Issue Tracker: If You See Something, Say Something
\end{itemize}
\begin{center}
{\bf Topic Czars}

\begin{itemize}
\item Gatekeeper for Contributions
\item Responsible for Setting Out Tests A Module Should Pass
\begin{itemize}
\item e.g.\ Special Cases With Analytical Solutions
\item Metrics for ``closeness'' to ``true'' solution
\end{itemize}
\end{itemize}

\pause 
\begin{tabular}{lll}
Name & Topic & Affiliation
\\ \hline  Serguei Maliar & Interpolation & Stanford %& \href{mailto:ccarroll@llorracc.org}{ccarroll@llorracc.org}
\\ Lilia Maliar & Interpolation & Stanford % & \href{}{mailto:ccarroll@llorracc.org}
\\  \multicolumn{3}{c}{{\it We're Seeking Volunteers for Czars}}
\end{tabular}
\end{center}

\pause
\begin{tabular}{rcl}
\hline \href{mailto:info@econ-ark.org}{info@econ-ark.org} & - & General Purpose Questions
\\ \href{mailto:czars@econ-ark.org}{czars@econ-ark.org} & - & Volunteer to be a Czar
\\ \href{mailto:ideas@econ-ark.org}{ideas@econ-ark.org} & - & Ideas for Improvement
\end{tabular}


\end{frame}

\subsection{When}

\begin{frame}
\frametitle{Timeline}

\begin{center}
\begin{tabular}{rll}
When & What & Lessons
\\ \hline 2006-2013 & \href{http://econ.jhu.edu/people/ccarroll/SolvingMicroDSOPs}{SolvingMicroDSOPs} & Surprisingly popular
\\ 2014-12 & \href{}{IMF-CFPB Workshop} & Lots of enthusiasm
\\ 2015-12 & \href{}{CFPB-IMF Workshop} & Not HARK, ARK!  
\\ & & Testing, Replication, Feedback
\\ 2016-06 & Hello! & None yet ...
\end{tabular}
\end{center}

\begin{itemize}
\item The version at \href{github.com/econ-ark}{http://github.com/econ-ark}  is our ``public beta''
\item So far as we know, everything works
\item First non-beta: Built-in tests for {\it everything}
\begin{itemize}
\item Aim: This year
\end{itemize}
\end{itemize}

\end{frame}

\subsection{Why?}
\begin{frame}
\frametitle{Why Are Policy Institutions So Interested?}

Participation: CFPB, OFR, IMF 

Interest From: FRB, ECB, BLS

\begin{itemize}
\item Policymaking = Applied Theory.  Options:
\begin{enumerate}
\item Informal, intuitive, ``wetware'' theory 
\item Formal, structural, ``software'' theory 
\end{enumerate}
\end{itemize}

\end{frame}

\begin{frame}
\frametitle{LATE is Antedeluvian}

`Local Average Treatment Effects' results are 
\begin{itemize}
  \item {\bf N}ot {\bf E}ven {\bf V}ery {\bf E}mpirically {\bf R}elevant ...
  \item UNLESS used to estimate `structural' parameters
\item Because the important question is
\begin{itemize}
\item What does world look like {\it non-locally} ...
\item ... = {\it after} the policy change
\item and maybe not even just ``on average''
\begin{itemize}
\item because distributional/targeted impact may be whole point
\end{itemize}
\end{itemize}

\end{itemize}

\end{frame}

\begin{frame}
\frametitle{Welfare Analysis With Heterogeneity}

Sensible cost-benefit analysis requires:
\begin{itemize}
\item Estimates of distribution of heterogeneous outcomes 
\item Utility or other weighting of those outcomes
\item $\rightarrow$ Structure
\end{itemize}

\end{frame}

\subsection{How}

\begin{frame}
\frametitle{{\it The Invention of Science} by David Wootton}

Wrong explanations for the Scientific Revolution: \pause 
\begin{itemize}
\item Invention of `the experiment'
\item Invention of the printing press
\item ... 
\end{itemize}
\medskip\medskip

\pause 
Right explanation:
\begin{itemize}
\item Creation of community of scholars
\item ... whose methods and results were `open source'
\item ... who critcized and improved and debugged each other
\end{itemize}

\medskip
\pause
Alchemy $\rightarrow$ Chemistry

\medskip
\pause 
17th and 18th century version of \href{http://github.com}{github.com}!

\end{frame}

\begin{frame}
\frametitle{Economists are People Too ...}

\begin{itemize}
\item We are {\it way} behind many scientific fields in `open source' code
\item Surveys/Experiments: Economics students are more `selfish.'
\item Options: \pause
\begin{enumerate}
\item `selfish' people study economics
\item Studying economics makes you selfish!
\item Economics students are just more honest
\end{enumerate}
\end{itemize}

\pause I prefer (3)!

\end{frame}

\begin{frame}
\frametitle{Lessons Learned from Other Fields About What Works}

\begin{itemize}
\item Not taking the dewy-eyed view: ``Build it and they will come''
\item Emprical fact: Many other open source communities have succeeded
\item Economists can't be {\it that} different ...
\end{itemize}

\end{frame}

\begin{frame}
\frametitle{In Addition to Usual Github Tools}

\begin{itemize}
\item Czars for specific topics
\item Bounties for Best Solution of Specific Problems
\item Time-Stamped Public Mechanism for Staking a Claim to New Idea
\item Stack-Exchange-Like Q\&A Forum
\item Mechanism for Easy Creation of Grad Student Problem Sets
\item Tool for Grad Student Replication Exercises
\item Eventually, a Journal?
\item ... Your Ideas?  \href{mailto:ideas@econ-ark.org}{ideas@econ-ark.org}
\end{itemize}

\end{frame}

\begin{frame}
\frametitle{Join our (Scientific) Revolution!}

\providecommand{\subscribe}{\href{mailto:subscribe@econ-ark.org}{subscribe@econ-ark.org}}
\providecommand{\letmehelpwith}{\href{mailto:letmehelpwith@econ-ark.org}{letmehelpwith@econ-ark.org}}
Options:
\begin{itemize}
\item \subscribe
\begin{itemize}
\item Add me to the newsletter/mailing list
\end{itemize}
\item {\it Read the docs and slides} and absorb what exists now.  Options:
\begin{enumerate}
\item Add an `issue' that you want to tackle on \ARK
\item \letmehelpwith
\begin{enumerate}
\item Define some area that you'd like to contribute to
\item email us at this address outlining what you propose to do
\item We'll reply with some suggestions
\end{enumerate}
\end{enumerate}
\end{itemize}

\end{frame}

\end{document}

